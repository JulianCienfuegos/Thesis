\begin{thebibliography}{1}
\bibitem{Amber} Amber Molecular Dynamics. \url{ambermd.org}
\bibitem{tessel} D. Austin. Voronoi diagrams and a day at the beach. \url{http://www.ams.org/samplings/feature-column/fcarc-voronoi}
\bibitem{3dwing} L. Bai et. al. 3D surface reconstruction and visualization of the Drosophila wing imaginal disk at cellular resolution. [Awaiting publication]
\bibitem{Yoshi} K. Bambardekar et. al. Direct Laser Manipulation Reveals the Mechanics of Cell Contacts In Vivo \emph{PNAS} \textbf{112} 1416-1421.
\bibitem{Boost}The C++ Boost Library.\url{www.boost.org}
\bibitem{rubr} Rubber stress-strain. \url{boundless.com}
\bibitem{Rubber} M. Boyce. Constitutive models of rubber elasticity: a review. \url{http://biomechanics.stanford.edu/me338/me338_project01.pdf}
\bibitem{MechanicalFeedback} Buchmann, A. Alber, M., Zartman, J. \emph{ Sizing it up: The mechanical feedback hypothesis of organ growth regulation.} Seminars in Cell and Developmental Biology, 2014.
\bibitem{CHARMM} CHARMM tutorial. \url{http://www.charmmtutorial.org/index.php/The_Energy_Function}
\bibitem{ChasteTutorial} \emph{Chaste Tutorial} \url{https://chaste.cs.ox.ac.uk/chaste/tutorials/release_2.1/UserTutorials.html}
\bibitem{epihyper2}V. Conte and JJ Munoz. A 3D finite element model of ventral furrow invagination in the Drosophila melanogaster embryo. \emph{J Mech Behav Biomed Mater.}\textbf{2} (188-198).
\bibitem{Drasdo} Drasdo, D. \emph{Bucking Instabilities of One Layered Growing Tissues.} Physical Review Letters 84.
\bibitem{DurandStone} Durand, M., Stone, H. Relaxation Time of the Topological T1 process in a Two Dimensional Foam. arXiv.0
\bibitem{Epithelium}\emph{Epithelial Tissues} \url{http://www.botany.uwc.ac.za/sci_ed/grade10/mammal/epithelial.htm}
\bibitem{Farhadifar} Farhadifar, R., Roper, J., Algouy, B., Eaton, S., Jullcher, F. The Influence of Cell Mechanics, Cell-Cell Interactions, and Proliferation on Epithelial Packing. \emph{Current Biology} \textbf{17} 2095-2104. (2007) 
\bibitem{VertexModels} Fletcher, A. , Osterfield, M., Baker, R., Shvartsman, S. \emph{Vertex Models of Epithelial Morphogenesis.} Biophysical Journal 106, June 2014.
\bibitem{ChasteMain} Fletcher, A., Osborne, J., Maini, P, Gavaghan, D. \emph{Implementing vertex dynamics models of cell populations in biology within a consitent computational framework.} Prog. Biophys. Mol. Biol. 113: 299 - 326.
\bibitem{DA} R. A. Fort and M. S. Steinberg. The differential adhesion hypothesis: a direct evaluation. \emph{Dev. Biol}\textbf{278} 255-263. (2005)
\bibitem{Giant} The Giant's Causeway Image. \url{http://images.nationalgeographic.com/wpf/media-live/photos/000/009/cache/giants-causeway_974_990x742.jpg}
\bibitem{Overview} Gibson, W., Gibson, M. \emph{Cell Topology, Geometry, and Morphogenesis  in Proliferating Epithelia.} Current Topics in Developmental Biology \textbf{89} 87-114. (2009)
\bibitem{Orientation} T. Gillies and C. Cabernard. Cell Division and Orientation in Animals. \emph{Current Biology} \textbf{21} 599-609
\bibitem{RelaxationofT1} P. Grassia, C. Oguey, R. Saepitomi. Relaxation of the topological T1 process in a two-dimensional foam. \emph{The European Physical Journal E} \textbf{35} (2012)
\bibitem{DirichletDomains} H. Honda. Description of Cellular Patterns by Dirichlet Domains: The Two-Dimensional Case. \emph{Journal of Theoretical Biology} \textbf{72} 523-543. (1978)
\bibitem{CellDivision} H. Honda, H. Yamanaka, M. Dan-Sohkawa. A Computer Simulation of Geometrical Configurations During Cell Division. \emph{Journal of Theoretical Biology} \textbf{106} 423-435. (1984)
\bibitem{Honda3D} H. Honda. A three-dimensional vertex dynamics cell model of space-filling polyhedra simulating cell behavior in a cell agregate. \emph{JTB} \textbf{226} 439-453 (2004)
\bibitem{ShapeFormation} Honda, H. Essence of Shape Formation in Animals. \emph{Forma}, \textbf{27} S1-S8. (2012)
\bibitem{q1} \url{Stackoverflow.org} ``How to construct a voronoi diagram inside a box?''
\bibitem{hyperbio2} F. Julicher et. al. Ective behavior  of the cytoskeleton. \emph{Nonequilibrium physics.}\textbf{449} (3-28)
\bibitem{1989Kawasaki} K. Kawasaki, T. Nagai, K. Nakashima. Vertex models for two-dimensional grain growth. \emph{Philisophical Magazine Part B} \textbf{60} 399 - 421. (1989)
\bibitem{LAMMPS} LAMMPS Tutorials. \url{http://lammps.sandia.gov/tutorials.html}
\bibitem{rose} K. Liu, S. Ernst, V. Lecaudey, O. Ronneberger. Epithelial Rosette Detection in Microscopic Images.
\bibitem{Udacity} Luebke, D. and Owens, J. \emph{Intro to Parallel Programming} \url{https://www.udacity.com/course/cs344} 2013.
\bibitem{CellSize} Marshall et. al. What Determines Cell Size? \emph{BMC Biology} \textbf{10} (2012)
\bibitem{Rubber2} A. Muhr. Modeling the Stress-Strain Behavior of Rubber.  \emph{Rubber Chemistry and Technology} \textbf{78}(391-425.)
\bibitem{Mao} Y. Mao. Differential proliferation rates generate patterns of mechanical tension that orient tissue growth. \emph{The EMBO Journal} \textbf{21} 2790-2803. (2013).
\bibitem{WoundHealing} T. Nagai and H. Honda. Wound Healing Mechanism in Epithelial Tissues Cell Adhesion to Basal Lamina. \emph{Proceedings of the 2006 WSEAS Int. Conf. on Cellular \& Molecular Biology, Biophysics \& Bioengineering}\textbf{2006} (pp111-116)
\bibitem{HondaNagai} Nagai, T. Honda, H. A dynamic model for the formation of epithelial tissues. \emph{Philisophical Magazine, Pt. B.}\textbf{81}(2001) 
\bibitem{VertDyn} T. Nagai, K. Kawasaki, K. Nakamura. Vertex dynamics models of two-dimensional cellular patterns. \emph{Journal of Physical Sciences Japan} \textbf{57} 2221-2224.
\bibitem{EpithelialTopology} R. Nagpal, A. Patel, M. Gibson. Epithelial Topology. \emph{BioEssays} \textbf{30} 260-266. (2008)
\bibitem{ScalingBehavior} K. Nakashima, T. Nagai, K. Kawasaki. Scaling Behavior of Two-Dimensional  Domain Growth: Computer Simulation of Vertex Models. \emph{Journal of Statistical Physics} \textbf{57} 759-787. (1989)
\bibitem{NAMD} NAMD Scalable Molecular Dynamics. \url{http://www.ks.uiuc.edu/Research/namd/}
\bibitem{hyperbio} A. Natali. Hyperelastic models for the analysis of soft tissue mechanics: definition of the constitutive parameters.\emph{Biomed. Rob. and BioMech}\textbf{2006}
\bibitem{TopStruct2DUnordered} Ohlenbusch, H.M., Aste, T. Dubertret, B., Rivier, N. The Topological Structure of 2D Disordered Cellular Structures. arXiv.
\bibitem{Okuda1} S.Okuda et. al. Reversible Network Reconnection Model for Simulating Large Deformation in Dynamic Tissue Morphogenesis. \emph{Biomech Model Mechanobiol} \textbf{12} 627-644 (2013)
\bibitem{NewOkuda} S. Okuda et. al. Coupling intercellular molecular signalling with multicellular deformation for simulating three-dimensional tissue morphogensis. \emph{Interface Focus} \textbf{5} (2015)
\bibitem{Okuda3}S. Okuda et. al.
\bibitem{OpenCV} Open Computer Vision. \url{opencv.org}.
\bibitem{WO}G. Oster and M. Weliky. The mechanical basis of cell rearrangement. \emph{Development} \textbf{109} 373-386. (1990)
\bibitem{epitheliumImage} \url{http://www.millerplace.k12.ny.us/webpages/lmiller/photos/636532/Epithelial\%20TissueTypes.bmp}
\bibitem{misaligned} J. Pease and J. Tirnauer. Mitotic spindle misorientation in cancer - out of alignment and into the fire. \emph{J. of Cell Science} \textbf{124} (1007-1016).
\bibitem{Order} K. Ragkousi and M. Gibson. Cell Division and the Maintenance of Epithelial Order. \emph{Journal of Cell Biology} \textbf{207} 181-188
\bibitem{Morphogen} Restrepo, S., Zartman, J. , and Basler, K. \emph{ Coordination of Patterning and Growth by the Morphogen DPP} Current Biology 24, 245- 255
\bibitem{Sokolow} A. Sokolow et. al. Cell Ingression and Apical Shape Oscillations during Dorsal Closure in \emph{Drosophila}. \textbf{102} 969-979.
\bibitem{TopologicalModels} R. Thom. Topological Models in Biology. \emph{Topology} \textbf{8} 313- 315 (1969)
\bibitem{triangle} Triangle: A two dimensional quality mesh generator. \url{https://www.cs.cmu.edu/~quake/triangle.html}
\bibitem{voro++} Voro++ - A 3D Voronoi Cell Software Library. \url{http://math.lbl.gov/voro++/}
\bibitem{Soap} Weaire, D. and Rivier, N. \emph{Soap, Cells, and Statistics}
\bibitem{Xiong} F. Xiong et. al. Interplay of Cell Shape and Division Orientation Promotes RObust Morphogenesis of Developing Epithelia. \emph{Cell} \textbf{2014} 415-427. (2014)
\bibitem{Checkers} H. Yamanaka and H. Honda. A checkerboard pattern manifested by the oviduct epithelium of the Japanese Quail. \emph{Int. J. Dev. Biol.} \textbf{34} 377-383.
\bibitem{epihyper} J. Zartman. Unit operations of tissue development: epithelial folding. \emph{Annu. Rev. Chem. Biomol. Eng. }\textbf{2010}
\end{thebibliography}


