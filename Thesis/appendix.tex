\appendix
\label{appendix}
\chapter{Getting, Running, and Modifying the Code} 
\section{GitHub}
When you are working on a large project such as this one, it is a good idea to have some sort of version control system which tracks the changes you have made to your code, and to return to an earlier working version in case something  gets terrribly broken. Many people who do not know about version control will do just this, except they will ´save as' every couple of days. Unfortunately, this method is very space inefficient, as each time you `save as', you save your entire project. Roughly speaking, git saves only the small changes you have made between versions. Git is a popular version control  tool, and github is a popular place to store you files online.

The following instructions show you how to create and clone git repositories. The repository for \emph{GrowFlesh} can be found at \url{https://github.com/JulianCienfuegos/NAGAIHONDAMODEL}.

\subsection*{Get Your Files on GitHub}
\begin{itemize}
\item Go to github.com and sign up for an account
\item Make a new repo on github
\item cd to the directory of your project on your machine.
\item git init
\item git add .
\item git commit -m ``some message"
\item git remote add origin YOUR URL HERE (This url is given to you from github when you make the repo.)
\item git pull origin master
\item git push origin master
\end{itemize}

\subsection*{Access And Modify These Files Somewhere Else}
\begin{itemize}
\item Simply type git clone \url{url of repository here} into a terminal
\item Work on project
\item git push origin master, when done working.
\end{itemize}

